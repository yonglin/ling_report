%\documentclass[journal]{IEEEtran}
\documentclass[12pt,onecolumn,a4paper]{IEEEtran}
\usepackage[utf8]{inputenc}
\usepackage{graphicx}
\usepackage{easylist}
\ifCLASSINFOpdf
\else

\fi

\hyphenation{op-tical net-works semi-conduc-tor}


\begin{document}
\linespread{1.5}
\title{The Second Year Internship Report}
\author{Ling~Wu\\
  Company Name: Hotel President Wilson\\
  Company Address: 47, Quai Wilson, Geneva, 1211, Switzerland\\
  From March 2014 to September 2014\\}
\maketitle

\IEEEpeerreviewmaketitle
\thispagestyle{empty}
\newpage

\tableofcontents
\thispagestyle{empty}

\newpage

\clearpage
\pagenumbering{arabic} 
\section{\textbf{Introduction}}

A pleasant internship.\\

The following report describes the activities carried out during a 12-week, full-time internship at the. The document contains information about the organization and the responsibilities performed throughout the period between June and August 2011. More than a plain account of tasks, the objective of this text is to reflect upon the experiences collected during the internship from the perspective of a MSc student in Sustainable Development.\\

The first part of the report offers an overview of the organization,followed by the working plan initially agreed upon with the Federation and approved by the University of Uppsala as a suitable internship. Following, it proceeds to describe in some detail the most relevant projects carried out and their respective analysis. Finally, the report wraps-up with a few closing remarks and conclusions from the experience.

\newpage
\section{\textbf{Description of the Establishment}}
Description of the establishment (No.staff, hierarchy, departments, market segment)\\

The following report describes the activities carried out during a 12-week, full-time internship at the. The document contains information about the organization and the responsibilities performed throughout the period between June and August 2011. More than a plain account of tasks, the objective of this text is to reflect upon the experiences collected during the internship from the perspective of a MSc student in
Sustainable Development.\\

The first part of the report offers an overview of the organization,followed by the working plan initially agreed upon with the Federation and approved by the University of Uppsala as a suitable internship. Following, it proceeds to describe in some detail the most relevant projects carried out and their respective analysis. Finally, the report
wraps-up with a few closing remarks and conclusions from the experience.

\newpage
\section{\textbf{Location of the Hotel}}
 region, town or village, city, country, surroundings\\
 
The following report describes the activities carried out during a 12-week, full-time internship at the. The document contains information about the organization and the responsibilities performed throughout the period between June and August 2011. More than a plain account of tasks, the objective of this text is to reflect upon the experiences collected during the internship from the perspective of a MSc student in
Sustainable Development.\\

The first part of the report offers an overview of the organization,followed by the working plan initially agreed upon with the Federation and approved by the University of Uppsala as a suitable internship. Following, it proceeds to describe in some detail the most relevant projects carried out and their respective analysis. Finally, the report
wraps-up with a few closing remarks and conclusions from the experience.

\newpage
\section{\textbf{Analysis of Different Competitors}}
The following report describes the activities carried out during a 12-week, full-time internship at the. The document contains information about the organization and the responsibilities performed throughout the period between June and August 2011. More than a plain account of tasks, the objective of this text is to reflect upon the experiences collected during the internship from the perspective of a MSc student in
Sustainable Development.\\

The first part of the report offers an overview of the organization,followed by the working plan initially agreed upon with the Federation and approved by the University of Uppsala as a suitable internship. Following, it proceeds to describe in some detail the most relevant projects carried out and their respective analysis. Finally, the report
wraps-up with a few closing remarks and conclusions from the experience.

\newpage
\section{\textbf{Responsibility In the Internship}}
include names and titles of people you worked with\\

The following report describes the activities carried out during a 12-week, full-time internship at the. The document contains information about the organization and the responsibilities performed throughout the period between June and August 2011. More than a plain account of tasks, the objective of this text is to reflect upon the experiences collected during the internship from the perspective of a MSc student in
Sustainable Development.\\

The first part of the report offers an overview of the organization,followed by the working plan initially agreed upon with the Federation and approved by the University of Uppsala as a suitable internship. Following, it proceeds to describe in some detail the most relevant projects carried out and their respective analysis. Finally, the report
wraps-up with a few closing remarks and conclusions from the experience.

\newpage
\section{\textbf{Working and Learning Experience}}
professionally and personally, negative and\\

The following report describes the activities carried out during a 12-week, full-time internship at the. The document contains information about the organization and the responsibilities performed throughout the period between June and August 2011. More than a plain account of tasks, the objective of this text is to reflect upon the experiences collected during the internship from the perspective of a MSc student in
Sustainable Development.\\

The first part of the report offers an overview of the organization,followed by the working plan initially agreed upon with the Federation and approved by the University of Uppsala as a suitable internship. Following, it proceeds to describe in some detail the most relevant projects carried out and their respective analysis. Finally, the report
wraps-up with a few closing remarks and conclusions from the experience.

\newpage
\section{\textbf{Case Study}}
It can be seen in the above report, I demonstrated the basic information and the organization structure of this luxuriant Hotel. At the same time, I discussed my role and duty in this internship and analyzed several other competitors as well. Then, without any doubt, I accumulated unprecedented professional experience personally. This internship left me with an indelible impression of the top Hospitality Industry, which motivates me to rethink something about my previous employer(\textbf{Hotel President Wilson}) from an employee's point of view. Since it was proved in Sam Glucksberg's Candle Experiment(1962) that conventional extrinsic encouragement like the bonus, would not work. 

In the case study section, I will focus on how \textbf{Hotel President Wilson} can stimulate employees' quality of service by appealing to their intrinsic motivation based on my observation. The rest of this case study is organized as follows: Subsection A outlines the historical psychological experiment ``Candle Experiment" and illustrates the significant importance of the intrinsic motivation. After that, the real experience of my internship is analyzed by implementing this ``candle model''. Subsection B presents that the ``Objectives and Key Results(OKR)'' method can be used for inspiring inherent motivation of the employee, which means higher efficiency and quality of services. In Subsection C, it is shown that some latest technologies needs to be introduced to promote our daily work such as well-designed schedule Apps with route optimization and some other artificial intelligence algorithms.

\subsection{\textbf{Candle Experiment}}
Some researchers consider the motivation of human beings as one of the greatest unsolved mysteries of ethology, psychology, and even economy. Since if we can figure out such a puzzle, we might expand our capability in numerous fields revolutionarily. Taking the business management as an example, as a manager, you can understand your employees' behaviors better, then you can arrange the most suitable people into the most proper positions. As an employee, you can make it clear which kind of work could motivate you better. Eventually, the whole organization will benefit from such great bilateral matching. 

Although the academic community did not believe that the intrinsic motivation can affect human beings more significantly than the extrinsic motivation until the Sam Gluckberg's``Candle Experiment'' 1962\cite{glucksberg}. Before this interesting behavior experiment, the mainstream point of view is that people can be inspired by the motivation from the outside rather than from the inside. 

Actually, the US psychologist Karl Duncker was first one to propose the candle problem or candle task in 1945. After that, in 1962, Sam Glucksberg\cite{glucksberg} redesigned the candle experiment in his research to present the influence of incentives. In Glucksberg's test, the participants were divided the participants into two groups and informed that they would be timed to monitor how long it would take when they tried to solve the candle problem.

\begin{itemize}
  \item For the first group, Glucksberg claimed that people who could addressed the Candle Problem most quickly would gain 20 dollars. Then the group member who was the 25\% fastest one would get a reward of 5 dollars.
  \item Turning to the second group, Glucksberg said that he was just trying to do an experiment to measure the average time that people would spend on solving such a problem significantly.
\end{itemize}

The result is as follows,
\begin{itemize}
  \item Normally, we might probably predict that the first group would finish the task more quickly, since they had considerable incentive . While the results turned out to be beyond our imagination. The first group spent three and a half minutes longer rather than the second group which was offered nothing. 
  
  \item The experiment stated the extrinsic motivation(here is the money) which was widely considered as the best tool to boost the productivity and the innovation, was just the wishful thinking. In addition to this, for the past 40 years, the result obtained from this experiment has been confirmed repeatedly.
\end{itemize}

\newpage
Now when it comes to our case, what could \textbf{Hotel President Wilson} learn from the Candle Problem? As my observation in my internship, I found that the Hotel offered the employees quit decent incentive including the bonuses, commissions, prizes. Taking me for instance, I can get 3000 Swiss Franks for each month which might be substantially high comparing to other companies. In addition to this, the management board held a surprise party for me when I was leaving back for Campus. Finally, I even got a five thousand Swiss Franks travel voucher, a nice Laptop, and an elegant hardcover \textit{\textbf{L'Étranger}} as a prize. 

I should say it seems impossible that I could get the better incentive in other competitors. Even though, I had such a feeling that I did not perform my own value and ability sufficiently well in my internship. The only thing I did over and over again was to follow the Hotel rules and the schedule, which make me like a good working machine not a creative human being. Although the hospitality service is quit a dynamic work which needs to be adjusted by the changes around real circumstance instantly. It would be amazing if I could have more flexibility and independence in my daily work, which can make me feel more satisfied and more successful rather than the higher payments or some other incentives. 

My own experience in \textbf{Hotel President Wilson} fits the Candle Experiment quit well, which suggests the greater extrinsic motivation does not mean much more to me comparing some other intrinsic desires such as the autonomy, the challenge, and the innovation. Hence if I were the manager, I should consider how to understand the employees' inherent motivation not just promote them by ``money". And in the \textbf{Subsection B Objectives and Key Results}, it will demonstrates some specific ways of how a manager can target the employees' intrinsic desires by so-called one-one meeting.

Does it come to an end for Sam Gluckberg's``Candle Experiment''? 

Actually, Glucksberg adjusted the experiment a little bit later and Dan Pink reexplained the latter version ``Candle Experiment'' at the TED Global Talk Event in July, 2009\cite{DanPink}. Here is the revised experiment.

Glucksberg reduced the difficulty. Then the solution seemed more obvious and direct, which meant the participants did not have to think deeply. The result was quit interesting that the first group would be offered money achieved the goal much more quickly than the control group.

Dan Pink\cite{DanPink} concluded that when people tried to do the mechanical things which did not need that creative thinking, the incentive could motivate their performance quit well rather than the intrinsic motivations. In addition, Dan Pink illustrated some specific positions\cite{DanPink} that needed more extrinsic motivation.

\begin{itemize}
  \item Certain kinds of accounting;
  \item Certain kinds of financial analysis;
  \item Certain kinds of computer programming, and so on.
\end{itemize}

Again, coming to our case, as a manager in the \textbf{Hotel President Wilson}, he or she needs to classify tasks as the mechanical-oriented and creative-oriented. Then the manager just needs to give more extrinsic motivation to the mechanical-oriented work and more intrinsic motivation to the creative-oriented assignment. To be more clear, I will specify the exact ways of how to utilize this result for the real organization problem in the next subsection.

\subsection{\textbf{Objectives and Key Results}}
Subsection A just tries to fit the case by using the ``Candle Experiment", but it does not give a very specific roadmap to implement the final results. Now it is time to introduce an useful tool or model called \textbf{Objectives and Key Results(OKR)}. 

The \textbf{Objectives and Key Results} is a great way of keeping good balance between the extrinsic and the intrinsic motivations, which becomes more and more popular all over the world. Especially, some internet giants like Google, Linkedin, and Facebook use it as their standard organization and management method. Although the OKR model originated from the high-tech company Intel, it is a general tool for business management. I think it can help \textbf{Hotel President Wilson} to address some management issues. 

At first, I will outline the basic background and idea of the OKR method. Before the OKR, there is a dominant performance measurement method called \textbf{Key Performance Indicator(KPI)}. The idea of KPI is top-down stream, the top management board needs to set the goal of the whole organization or company. After that, the organization goal will be divided into smaller ones until reaching each specific employee. Then the personal goal will be quantified as several performance indicators, which guarantee the management board estimate the performance of each level in the organization. As it can be seen that once the KPI is set, it will be operated strictly, which means it will sacrifice the flexibility and might harm the whole organization. 

Corresponding to the pitfalls of KPI, OKR has a quit different way of measuring the performance from the top level to the bottom level of a company. In May, 2013, Rick Klau the Google Ventures partner gave a speech ``How Google sets goals: OKRs"\cite{OKRGV} about the importance and the implementation of setting the objectives and the key results at Google from 1999. 

The following is the ``\textbf{Elements of an OKR}" presented in the Rick's slides\cite{OKRGV}.\\

\begin{enumerate}
  \item The Objective\ldots
  \begin{enumerate}
    \item is ambitious
    \item feels a tad uncomfortable
  \end{enumerate}
 \item The Key Results
  \begin{enumerate}
    \item clearly make the objective achievable
    \item are quantifiable
    \item lead to objective grading
  \end{enumerate}
\end{enumerate}

A whole OKR process starts from setting a objective which needs to be some kind of ambitious. Then in order to finish the objective, some specific and quantifiable results needs to be achieve. Here we just have to focus on the objective and can modify the key results based on changes of the circumstance. Obviously, OKR gives us more freedom and flexibility, which is quit suitable for the dynamic and creative work. While how can we apply the OKR in our daily job? Rick also demonstrated ``\textbf{Keys to OKRs}" in his talk\cite{OKRGV}.\\

\begin{enumerate}
  \item Keys to OKRs\cite{OKRGV}:
  \begin{enumerate}
    \item set quarterly and annually
    \item measurable
    \item set as personal, team, and company levels
    \item publicly available to the entire company
    \item graded each quarter
  \end{enumerate}
\end{enumerate}

In his slides, Rick directed the OKRs for the different levels\cite{OKRGV}.
\begin{enumerate}
  \item Personal/Team/Company\cite{OKRGV}:
  \begin{enumerate}
    \item Personal OKRs define what the person is working on
    \item Team OKRs define priorities for the team, not just the collection of all individual OKRs
    \item Company OKRs are big picture, top level focus for the entire company
  \end{enumerate}
\end{enumerate}

Based on the above knowledge, I will take my daily work in this internship as an example to illustrate a whole OKRs process. At \textbf{Hotel President Wilson}, one of my major task was to organize the banquet as the manager assistant for the \textbf{SOCIETE NAUTIQUE DE GENEVE} which held activities frequently in the last summer. As I mentioned in the Subsection A, the banquet service is quit dynamic, which asks the host staff to be more flexible and creative. Sometimes, we did not have time to think the established plan or report to the superior. Now we are going to define a OKRs process to control the service quality of organizing the banquet.

\newpage

Personal OKRs for holding the banquet:

\begin{enumerate}
  \item Objective: Successful banquets
  \item Key Results:
  \begin{enumerate}
    \item Complaint Rate under 0.05 monthly
    \item Satisfaction Rate above 0.9 monthly
    \item Absence Rate 0 monthly
  \end{enumerate}
\end{enumerate}

At the same time, the manger needs to arrange one-one talk with his or her employees. It is so important that the manager can find who is proper for the creative work and who is good at doing the repeated job. Then the human resource of the whole team can be optimized.\\ 

At last, the Hotel is supposed to grade the hotel OKRs as follows:
\begin{enumerate}
  \item Hotel-wide monthly meeting\cite{OKRGV}
  \begin{enumerate}
    \item grade the last month's OKRs
    \item get OKR owner to explain the grade, explain adjustments for upcoming month
    \item set this month's OKR
  \end{enumerate}
\end{enumerate}

By applying such a process, it would be convinced that we just care about an employee's affect on a task not the effort on a job. Since the OKRs guarantee us more flexibility to adjust the ``Key Results" in order to achieve the main ``Objective". I think the service for the \textbf{SOCIETE NAUTIQUE DE GENEVE} would be improved if we can introduce such OKRs technique in our daily work.

\subsection{\textbf{New Technology Trend}}
We have linked two management models with each other in the previous subsections. Besides the novel management theory, it is stated that new technologies can always boost our productivity. In this subsection, I want to discuss two technologies which might be used in the Hotel. 

The first one is the well-designed and highly customized schedule App for smartphone. In our daily work, we often have to modify the schedule to fit the customer's demand, which will cause a serious issue that we cannot inform all of colleagues instantly. Imagining if the organizer can update the latest schedule on their smart phones, then the notification will instantly synchronize to each participant's mobile device no matter he or she is the customer or the Hotel employee. It will improve our service level dramatically.

The second recommended new technology is a tracing App which monitors the servants walking routes. By collecting the data, some optimization algorithms can be used to choose the shortest route for our employees according to the real-time situation in the banquet place, which can save time a lot.

I think this is just the beginning of combining the latest high-tech stuff with the traditional Hospitality industry. In the future, the customers' special behaviors might be recognized by analyzing their record in the Hotel. And it is also possible that the HR department could automatically get a customized motivation plan for each employee by mining their personal professional data. It seems that new technologies will reshape our Hospitality industry eventually. 
\newpage
\section{\textbf{Conclusion}}


% \ifCLASSOPTIONcaptionsoff
%   \newpage
% \fi
\newpage


\renewcommand\refname{Reference}
\small
\bibliographystyle{IEEEtran}
\bibliography{Bib}


\end{document}
